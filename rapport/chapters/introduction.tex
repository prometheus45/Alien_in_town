\chapter*{Introduction}
\addcontentsline{toc}{chapter}{Introduction}

\section{Presentation du jeu}

\subsection{Le jeu en presque 150 mots}
\subparagraph{}
\textnormal{ Le jeu s'appellera Alien in town.\
Le principe du jeu venant de deux jeux assez proches :}
\begin{itemize}
\item \textnormal{Mafia, present sur Starcraft II}
\item \textnormal{Loup-garou de Thiercelieux.}
\end{itemize}
\textnormal{ Dans ces deux jeux, deux équipes s'affrontent pour la victoire. \
Dans le premier il s'agit de mafieux et de villageois, dans le second de loups-garous et de villageois.
Pour continuer dans l'optique du jeu, il s'agira pour nous d'aliens et de citoyens (ville ou village).
Outre quelques exceptions, le but du jeu est de supprimer tous les membres de l'équipe adverse pour s'assurer la \
victoire.
Le système fonctionne sur une question de jour et de nuit.\
Les aliens ayant la possibilité de supprimer un citoyen
pendant la nuit et les citoyens de supprimer quelqu'un pendant le jour.\
L'intérêt du jeu étant de se cacher des citoyens pour les aliens et de trouver les aliens pour les citoyens.}

\subsection{Le jeu en presque 1000 mots}
\subparagraph{Origine}
\textnormal{ Le jeu s'appellera Alien in town.\
Le principe du jeu venant de deux jeux assez proches.}
\begin{itemize}
\item \textnormal{Mafia, present sur Starcraft II}
\item \textnormal{Loup-garou de Thiercelieux.}
\end{itemize}
\subparagraph{Les similitudes avec les jeux d'origines}
\textnormal{ Dans Alien in town comme dans les jeux d'origines:}
\begin{itemize}
\item \textnormal{Deux équipes de races difféntes cherchent a tuer l'équipe adverse.}
\item \textnormal{Le système fonctionne sur une question de jour et de nuit.}
\item \textnormal{Une équipe aura un chat de nuit et s'attaquera à l'autre équipe de nuit.}
\item \textnormal{L'autre équipe pourra décider par concensus de supprimer une personne.}
\item \textnormal{Les deux équipes auront un chat de jour.}
\item \textnormal{Des rôles particuliers existeront dans les deux camps.}
\end{itemize}
\subparagraph{Les différences avec les jeux d'origines}
\begin{itemize}
\item \textnormal{La plateforme est un smartphone.}
\item \textnormal{L'équipe qui donne le nom au jeu sera }
\item \textnormal{Une équipe aura un chat de nuit et s'attaquera à l'autre équipe de nuit.}
\item \textnormal{L'autre équipe pourra décider par concensus de supprimer une personne.}
\item \textnormal{Les deux équipes auront un chat de jour.}
\item \textnormal{Des rôles particuliers existeront dans les deux camps.}
\end{itemize}


